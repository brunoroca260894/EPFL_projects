%%%
% Computer Simulation of Physical Systems I (A. Pasquarello, EPFL)
% Outline of Task II
% -- author: Gabriele Sclauzero (EPFL) -- October 2010
% -- update: Alexey Tal (EPFL) -- May 2021
%%%
\documentclass[12pt]{article}

\usepackage[text={15.5cm,24.5cm},centering]{geometry}
\setlength{\parindent}{0mm}
\setlength{\parskip}{6pt}
\pagestyle{empty}
\usepackage{hyperref}
\usepackage{amssymb}

\usepackage{xcolor}
\usepackage{times}
\usepackage{amsmath}
\usepackage{listings}
%\usepackage{epsfig}
%\renewcommand{\labelenumi}{\textbf{\theenumi.}}
\usepackage{enumitem}
  \setenumerate{leftmargin=*,font=\bfseries}
  \setitemize{leftmargin=*,font=\bfseries} \definecolor{codegreen}{rgb}{0,0.6,0}
  \definecolor{codegray}{rgb}{0.5,0.5,0.5}
  \definecolor{codepurple}{rgb}{0.58,0,0.82}
  \definecolor{backcolour}{rgb}{0.95,0.95,0.92}
 
  \lstdefinestyle{mystyle}{
	  backgroundcolor=\color{backcolour},   
	  commentstyle=\color{codegreen},
	  keywordstyle=\color{magenta},
	  numberstyle=\tiny\color{codegray},
	  stringstyle=\color{codepurple},
	  basicstyle=\ttfamily\footnotesize,
	  breakatwhitespace=false,         
	  breaklines=true,                 
	  captionpos=b,                    
	  keepspaces=true,                 
	  numbers=none,                    
	  numbersep=5pt,                  
	  showspaces=false,                
	  showstringspaces=false,
	  showtabs=false,                  
	  tabsize=2
  }
  \lstset{style=mystyle}
	\hypersetup{
	    colorlinks=true,
			linkcolor=blue,
	    }



\newcommand{\bo}[1]{\boldsymbol{#1}}
\newcommand{\ee}{\ensuremath{\varepsilon}}
\newcommand{\sg}{\ensuremath{\sigma}}
\newcommand{\ang}{\ensuremath{\textnormal{\AA}}}
\newcommand{\secs}{\ensuremath{\textrm{s}}}



\begin{document}
\begin{center}
\rule[2mm]{13.5cm}{.1pt} \\
{\Large {\bf Computer simulation of physical systems I}}\\
\rule[2mm]{13.5cm}{.1pt}
\end{center}
\vspace{-1cm}
\begin{center}
{\bf Task II: NVE molecular dynamics simulations} \\
\end{center}

In this exercise, you are going to perform molecular dynamics (MD) simulations
of a system of atoms interacting through a Lennard-Jones pair potential in the
NVE statistical ensemble. To complete this exercise you should download the
python version of the MD code available on the website and closely follow the
instructions provided in this guide. 

All the functions necessary for performing MD simulations are implemented in
\verb!MD.py!, which has to be imported in your python code, i.e.
\verb!from MD import *!.

\begin{enumerate}
  \item \emph{Startup}: generate the starting atomic positions, the starting
	  velocities, then change the system temperature to the desired value
	  and equilibrate the sample.
  \begin{itemize}
		\item Use the function \verb!crystal()! to generate
	  atomic positions on a FCC lattice; choose the parameters to match
	  the number of atoms ($N=864$) and density
	  ($\rho=1.374\,\mathrm{g\cdot cm^{-3}}$) for liquid Ar as in Rahman's
	  paper\footnote{A. Rahman, Phys. Rev. A \textbf{136}, A405 (1964).}
	  (see also additional notes).
    \item Perform a short run to compute the velocities 
						and then use the constant velocity rescaling to adjust
	  the temperature to the value that you can find in Rahman's paper
	  ($T\simeq94\,\textrm{K}$).
    \item Equilibrate the system with a run in free evolution mode (regular NVE
	  dynamics); monitor the conservation of energy and the fluctuations
	  of other quantities.
    \item (Optional): check the conservation of energy as a function of the
	  integration step $\Delta t$; study the dependency of the
	  fluctuations on the system size (i.e., as a function of $N$).
  \end{itemize}

\item \emph{Sample static properties}: sample the radial pair correlation
	function, $g(r)$, and also the structure factor, $S(k)$, computed as the
	radial Fourier Transform (FT) of $g(r)$. Compare the position of the
	peaks with those reported in Rahman's paper. Try to explain the
	behaviour of $g(r)$ and its limits for $r\to0$ and $r\to\infty$.

\item \emph{Sample dynamical properties}: estimate the diffusion coefficient,
	$D$, using two different methods, namely
  \begin{enumerate}
    \item Einstein's relation, $\textrm{MSD}(t) = C + 6 D t$, which involves the
	    sampling of the mean square displacement (MSD) and a fitting
	    procedure giving $C$ and $D$
    \item the time integral of the velocity autocorrelation function 
   \end{enumerate}

\item (Optional) Repeat the same steps for a system at a different temperature
	and/or density (remember that the accuracy of the time integration may
	depend on these parameters) and compare the static and dynamic
	properties. 
\end{enumerate}


\clearpage
\begin{center}
\bf Task II: additional notes on MD units 
\end{center}

The MD codes provided for this exercise (and the next one) solve numerically the
equations of motion for a Lennard-Jones (LJ) fluid made of $N$ particles
contained in a 3D periodically repeated rectangular box and interacting through
the pair potential:
\[
V(r) = 4\ee \left[ \left(\frac{\sg}{r}\right)^{12}
 -\left(\frac{\sg}{r}\right)^6\right] \;.
\]
The codes use internally the so-called LJ units (i.e. \sg\ and \ee\ are both set
to $1$), so that distances are expressed in \sg\ units, while energies are given
in \ee\ units.  Therefore, once the parameters of the the LJ potential have been
fixed (for instance, $\sg=3.4\,\ang$ and $\ee/k_B = 120\,K$ for Ar, see Rahman's
paper) the units of measure for all physical quantities are also fixed, as
listed in the following table for those of our interest:
\begin{center}
 \begin{tabular}{c|c}
quantity & LJ units \\
\hline
distance & \sg\ \\
energy  & \ee\ \\
velocity & $(\ee/M)^{1/2}$ \\
temperature & $\ee/k_B$ \\
time & $\sg (\ee/M)^{-1/2}$ \\
 \end{tabular}
\end{center}
where $M$ is the mass of an atom in the simulation (for instance, $M \simeq 6.69
\cdot 10^{-26}$ kg for Ar). 

As an example, we can use the relations above to convert the integration time
step used in Rahman's paper ($\Delta t = 10^{-14} \secs$) from seconds to LJ
units. Inverting the relation 
\[
\Delta t = \sg (\ee/M)^{-1/2} \cdot \Delta u
\]
to obtain $\Delta u$ (in LJ units), we get the following value (expressing all
parameters in SI units):
\[
\Delta u =  \Delta t \cdot (\ee/M)^{1/2}/\sg = 10^{-14} \cdot \left(\frac{120
\cdot 1.38 \cdot 10^{-23}}{6.69 \cdot 10^{-26}}\right)^{1/2} \cdot \frac{1}{3.4
\cdot 10^{-10}} \simeq 0.0046\;,
\]
which will be employed in our simulations in order to compare with Rahman's results.

\newpage
%---------------------------------------------------------------------------------
\section*{Instructions}
%---------------------------------------------------------------------------------

\begin{itemize}
\item Download and unpack the archive \verb!Task2.zip!. The MD code is
implemented in \verb!MD.py! and the files \verb!Step*.py! contain minimal scripts
required to complete the step.


\item  if you managed to get the code working on your workstation, then you are
ready to start: change directory to \verb!Step1_Startup! and move to Step 1.
\end{itemize}

%---------------------------------------------------------------------------------
\subsection*{MD code}
%---------------------------------------------------------------------------------
The MD code requires the following libraries: 
\begin{itemize}
\item \verb!numpy! for linear algebra (\href{https://numpy.org}{https://numpy.org})
\item	\verb!matplotlib! for plotting (\href{https://matplotlib.org}{https://matplotlib.org})
\item	\verb!scipy! for statistics analysis and fitting (\href{https://scipy.org}{https://scipy.org})
\item	\verb!numba! for acceleration of the most heavy functions
(\href{http://numba.pydata.org}{http://numba.pinata.org}). The use of 
this library can be easily avoided if necessary, but be prepared to loose the performance.
To eliminate \verb!numba! from the code, remove the line
\verb!@jit(parallel=True)! In \verb!MD.py!.
\end{itemize}

The main functions used for running MD simulation are \verb!run_NVE()! and
\verb!run_NVT()!. These functions return all the results in a dictionary, which
contains energies, velocities, $g(r)$ etc. You should find all the possible
output in the return statement of the corresponding function.



%---------------------------------------------------------------------------------
\subsection*{Step 1: Startup}
%---------------------------------------------------------------------------------
\subsubsection*{1. Running \texttt{crystal()}}
First of all, you must generate a configuration to start from.
The function \verb!crystal()! is used to arrange atoms in a crystalline fcc
structure.

The function \verb!crystral()! takes two arguments:
the number of units fcc cells along each direction \verb!Ncells! and 
the lattice spacing \verb!lat_par!.

The number of unit fcc cells (containing 4 atoms each) to stack along the
three directions: choose them in order to get a cubic box with same number of
particles ($N=864$) used in [1], hence select 6 unit cells along each axis so
that N will be equal to $4 \times (6 \times 6 \times 6) = 864$ (in general you
should not put less unit cells than what suggested to satisfy the minimum image
criterion, but 6 cells is more than enough in this example).  This number of
cells, combined with the lattice parameter chosen above, gives a box size
approximately equal to that in [1] ($L=10.229$ in L.J. units, please see the
notes), so that the densities will be the same too.

The lattice spacing of the fcc crystal is the equilibrium lattice spacing of
the LJ potential is $1.5496$, but here we choose a value, $a=1.7048$, that
corresponds to the density studied by Rahman [1], i.e.  $1.374~\mathrm{g\cdot
cm}^{-3}$ for Ar (with atomic mass approx.  $M = 6.69\cdot10^{-23} \mathrm{g}$).

The function \verb!crystal()! returns two arrays: coordinates and velocities, the
latter assigned randomly according to Gaussian distribution.

The simplest example of using \verb!crystal()! is provided in \verb!Step1.py!:
\begin{lstlisting}[language=python]
Ncells = 6          # Number of unit cells along each axis
lat_par = 1.7048    # Lattice parameter
L = lat_par*Ncells  # Size of the simulation box
N = 4*Ncells**3     # Number of atoms in the simulation box

# Generate fcc structure
pos, vel = crystal(Ncells, lat_par)
\end{lstlisting}



\subsubsection*{2. Running the MD code}
In order to run the MD code, you need to call \verb!run_NVE()! which takes six
compulsory arguments: coordinates, velocities, box size, \# steps, \# atoms, integration step.

For example, the simplest script could look like this:

\begin{lstlisting}[language=python]
Ncells = 6          # Number of unit cells along each axis
lat_par = 1.7048    # Lattice parameter
L = lat_par*Ncells  # Size of the simulation box
N = 4*Ncells**3     # Number of atoms in the simulation box
nsteps = 200        # Number of steps
dt = 0.003          # Integration step

# Generate fcc structure
pos, vel = crystal(Ncells, lat_par)

# Perform simulation and collect the output into a dictionary
output = run_NVE(pos, vel, L, nsteps, N, dt)
\end{lstlisting}

Files \verb!Step*.py! contain a minimal setup needed to complete each step. In
order to perform one step at a time make sure to comment the corresponding
part of the code (steps are separated by descriptive comments).

To run MD with output on the screen:
	\verb!python Step1.py!

To run MD with output on a file:
	\verb!python Step1.py > testrun.out!

To run MD with output on both file and screen:
	\verb!python Step1.py | tee testrun.out !


The first part of \verb!Step1.py! will perform a constant energy calculation
(NVE ensemble) with 200 steps (using a time step of 0.003), continuing from
\verb!sample10.dat!  previously generated (or created by crystal), and writing
on \verb!sample11.dat!  at the end.

On standard output (or inside \verb!testrun.out!) you will find 
some important quantities monitored at each time step,
such as kinetic and potential energies. 

\subsubsection*{3. Compute velocities}

In order to bring the sample close to the desired temperature (through
constant velocity rescaling), we first need to compute the velocities 
for the atomic configuration generated with crystal. A small number of 
time steps (here, 200) is sufficient for this purpose.

The input file is stored as \verb!md_start.in!:

\begin{lstlisting}[language=python]
nsteps = 200        # Number of steps
dt = 0.003          # Integration step

# Read crystal shape, positions and velocities from a file
N, L, pos, vel = read_pos_vel('sample10.dat')

# Perform simulation and collect the output into a dictionary
output = run_NVE(pos, vel, L, nsteps, N, dt)
\end{lstlisting}

For this tutorial, we will adopt an integration time step corresponding
approximately to that used in [1] for liquid Ar ($10^{-14}~\mathrm{sec}$., see
notes for the conversion to L.J. units). Among other things, you will be asked
to check how your results depend on the time step: the value needed to ensure
conservation of energy to a good extent depends on the temperature and on the
particle density.

\subsubsection*{4. Change T and equilibrate}

Now we are ready to apply the constant velocity rescaling to our sample: at
each time step the velocities will be scaled in order to bring the
instantaneous temperature of the system to the desired value ($T=94.4\mathrm{K}$, which
corresponds to about $0.7867$ in L.J. units for Ar).

The input is:

\begin{lstlisting}[language=python]
nsteps = 200
dt = 0.0046
T = 0.7867          # requested temperature

# Change T 
output = run_NVE(output['pos'], output['vel'], L, nsteps, N, dt, T)

# Plot temperature vs step
plt.plot(output['nsteps'],output['EnKin']*2/3)
plt.show()
\end{lstlisting}

T is an optional argument of the function \verb!run_NVE!, which default value is \verb!None!.
When T is greater than or equal to 0, the code
will run a run at constant temperature. Notice that is NOT a constant energy
dynamics, hence we are not sampling the NVE ensemble during this run (nor the
NVT ensemble, see Task3 for NVT molecular dynamics). Since we are interested 
in the equilibrium properties (in the thermodynamics sense) of the system, 
no data should be collected in this kind of run, however you can see how the
temperature changes during the run by plotting it against the step.

Before starting to collect data, we need to equilibrate the sample with a short
run of regular NVE dynamics. 

The input will be as follows:
\begin{lstlisting}[language=python]
nsteps = 800
dt = 0.0046

# Equilibrate
output = run_NVE(output['pos'], output['vel'], L, nsteps, N, dt)

# Plot total energy vs step
plt.plot(output['nsteps'],output['EnKin']+output['EnPot'])
plt.show()
\end{lstlisting}

By plotting the total energy, as a function of time you can check that 
$E_{tot}$ is actually conserved (to a good approximation) in this kind of dynamics
(and compare with what happens to $E_{tot}$ in the constant velocity rescaling run).
You can verify that the conservation of energy becomes more strict as the time
step 'deltat' is reduced.
In general, the other quantities display much larger fluctuations, instead. 
Notice that the average temperature might not be equal or not even close to the
target temperature, since in the NVE dynamics is not possible to fix this
variable (sometimes this makes also difficult to compare different MD 
simulations).


%---------------------------------------------------------------------------------
\subsection*{Step 2: Sample static properties}
%---------------------------------------------------------------------------------
\subsubsection*{1. Compute $g(r)$ and $S(k)$ (through F.T.)}

From the previously equilibrated atomic sample (which should be now stored in \\
\verb!sampleT94.4.dat!) you can start a MD run in which you do a sampling
of some physical properties.
We will first focus on some static properties, namely the radial pair
correlation function $g(r)$ and the structure factor $S(k)$. The latter can be
obtained in two modes, either directly by sampling the Fourier transform (FT)
of the number density, or, in the case of an isotropic system, as the FT of
the pair correlation function (see notes and Allen-Tildesly, ch. 2.6). In this
subtask you will proceed through the second way.


The code can be used to perform a MD run of 2000
steps and evaluate the $g(r)$ at every step. The quantity is then averaged over
all these samplings.

The code for $g(r)$ and $S(k)$ (through FT of $g(r)$) sampling is stored in \verb!Step2.py!:
\begin{lstlisting}[language=python]
nsteps = 2000
dt = 0.0046
N, L, pos, vel = read_pos_vel('sampleT94.4.dat')

# Run MD simulation
output = run_NVE(pos, vel, L, nsteps, N, dt)

# Plot g(r)
plt.plot(output['gofr']['r'],output['gofr']['g'])
plt.show()

# Plot S(k)
plt.plot(output['sofk']['k'],output['sofk']['s'])
plt.show()
\end{lstlisting}


\paragraph{TO DO:}
\begin{enumerate}
\item Measure position of the peaks (both in $g(r)$ and in $S(k)$) and compare to
those reported by Rahman). Try to explain the other features you see.

\item Study the behaviour of these two quantities as a function of the
equilibration temperature and of the density.
For the former, you need to go through the steps seen before in order to 
bring the system close to the new temperature and equilibrate. For the
latter you have either to generate a new sample with \verb!cyrstal()!.

\item You may try a simulation with a larger number of atoms in order to extend
the maximum radius allowed for $g(r)$, which is here limited to half of the
box size (see next lectures for other methods to extend this limit). 
Be aware that when the number of atoms gets larger than a few thousands the
code will become quite slow (due to the $O(N^2))$ operations. In order to
overcome this you may have to use another version of the code which uses
Verlet neighbor lists (at least for the dynamical evolution part).

\item When dealing with short range interactions (such as the LJ pair potential),
the potential is approximated by truncating and setting it to a fixed value
for interparticle distances beyond a certain cutoff radius (called
\verb!r_cutoff!  in the code). By changing \verb!r_cutoff! from its default value (2.5), 
you can check if and how this approximation affects the structural properties.  
\end{enumerate}


%---------------------------------------------------------------------------------
\subsection*{Step 3: Sample dynamical properties}
%---------------------------------------------------------------------------------
\subsubsection*{Compute MSD and VACF}

Now we move to the study of a dynamical quantity which is easily accessible
through MD simulations: the diffusion coefficient. As you have learned during
the class, this quantity can be computed from the mean square displacement
(MSD) of the atomic positions through Einstein's relation, or from the integral
of the velocity autocorrelation function (VACF).


\paragraph{Important remarks}
\begin{enumerate}
\item ensemble average needs either to average on different time origins in the
same run (not implemented) or to average (at same times) the quantity obtained
as a function of time in several different runs (of same length).
Smaller systems (small N) are subject to larger statistical fluctuations, 
therefore the deviation from the ensamble average may be quite large if a
single realization is used to estimate the diffusion coefficient.

\item since we are sampling a dynamical quantity, the accuracy in the description
of the particle trajectories plays more important role here. Be careful on the
choice of \verb!dt!; the value used so far may not be sufficiently small.
\end{enumerate}

\verb!Step3.py! adopts the second method for computing the ensemble average of the
dynamical quantities, i.e. the MSD and VACF are calculated as an average over a number
of simulations defined by \verb!Nruns!. 


\end{document}
